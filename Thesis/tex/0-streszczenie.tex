\streszczenie

Niniejsza praca miała na celu kontynuację prac prowadzonych nad sztuczną skórą na wydziale EiTI w ramach projektu INCARE pt. „Zintegrowany system innowacyjnych rozwiązań dla opieki nad osobami starszymi", finansowanego przez Narodowe Centrum Badan i Rozwoju na podstawie umowy nr AL2/2/INCARE/2018. Celem było stworzenie nowego prototypu sztucznej skóry i jego integracja na robocie asystującym Tiago działającym pod kontrolą systemu ROS. Zbudowany prototyp został zamocowany wokół robota, aby chronić go przed wywieraniem na niego nacisku powodowanego przez otoczenie lub człowieka, i odjeżdżać od jego źródła.

Praca opisuje zastosowane podejście technologii wykonania sztucznej skóry w porównaniu do podejścia innych zespołów badawczych ze świata. Opisany został szczegółowo proces projektowania prototypu sztucznej skóry oraz podjęte decyzje od budowy mechanicznej, przez projekt części elektronicznej i wykonywanych przez nią zadań, aż po część programistyczną integrującą sztuczną skórę z systemem sterowania robotem. Znajduje się tu także rozdział poświęcony badaniom nad doborem odpowiedniego materiału nośnego i ochronnego dla projektowanego rozwiązania. Gotowe rozwiązanie zostało również pomyślnie przetestowane w symulacji, w programie Gazebo oraz w laboratorium na rzeczywistym robocie.
\slowakluczowe sztuczna skóra, robot asystujący, Tiago, ROS

%--------------------------------------
% Streszczenie po angielsku
%--------------------------------------
\newpage
\abstract

This thesis was intended to continue the work on artificial skin developed at the EiTI department within the INCARE AAL-2017-059 project ,,Integrated Solution for Innovative Elderly Care'' by the AAL JP and co-funded by the AAL JP countries (National Centre for Research and Development, Poland under Grant AAL2/2/INCARE/2018). The aim was to create a new artificial skin prototype and integrate it with the Tiago service robot, working on the ROS system. The built prototype was attached around the robot in order to protect robot from the impact caused by the environment or the human and to drive away from its source.

The thesis describes the approach used to manufacture the artificial skin in comparison to other research teams' approaches from around the world. It describes in detail the design process of the artificial skin prototype and the decisions made from the mechanical construction, through the design of the electronic part and the tasks it handles, to the programming part that integrates the artificial skin with the robot control system. There is also a chapter dedicated to research on the selection of a suitable carrier and protective material for the designed solution. The finished solution has also been successfully tested in simulation, in the Gazebo software and in the laboratory on the real robot.

\keywords artificial skin, assistive robot, Tiago, ROS
